% -*- coding=utf-8 -*-
%  模版标题——调研报告(这部分无需修改)
\begin{center}
	{\zihao{3} \heiti 调研报告} 
\end{center}
\vspace*{\baselineskip}

%===================模版操作区===========================

调研报告正文应包括:选题的名称、来源及意义,国内外发展状况,国内外研究(应用)现状综述,本题目的研究(设计)目标、研究(设计)内容、研究(设计)方法、研究(设计)方案的可行性分析和已具备的条件,进度安排,主要参考文献等。

排版格式:正文小四号宋体,行间距20磅(图、表用“单倍行间距”);各级标题居左,正文各自然段开头空两字格;一级标题四号黑体,二级标题小四号黑体。

调研报告不少于5000字。\cite{王亚平2006}\cite{Sjostrand:2006za}

以下内容为调研报告内容参考

\section{来源与意义}
主要内容:课题研究的时代背景、政策背景与意义(1000字左右)

\subsection{选题背景}
选题的具体时代背景要求,国家或地方的政策支持,行业的发展前沿趋势

\subsection{选题意义}
选题意义,研究的现实意义和实用价值

\section{国内外的发展和研究(应用)现状}

发展现状和研究(应用)现状(2000字左右)

\subsection{调研说明}
调研说明(现场调查、问卷调查,资料查阅)

\subsection{现有管理方式}
现有的管理方式(业务流程)、系统应用情况

\subsection{国内外研究成果}
选题侧重算法或技术的,应包括国内外专家学者对该领域的研究成果和新技术应用等内容

\subsection{存在的问题}
问题说明

\section{研究内容}
系统分析(设计)与实现的具体内容(1500字左右)

\subsection{系统实现目标}
系统实现目标(结合第二部分的内容,提出可实现的目标)

\subsection{设计实现内容}
设计实现内容(结合课题背景,业务建模、系统需求、分析设计)

\subsection{应用技术}
应用技术(系统设计应用的软硬件技术,开发平台、架构、数据库等)

\subsection{进度安排}

\section{可行性分析}
可行性分析和已具备条件(500字左右)

\subsection{本专业课程支撑}
本专业课程支撑

\subsection{校内资源}
校园网络数据资源、图书资料和网络在线学习等资源

\subsection{硬件设施}
用于开发的硬件设施(设备)和开发工具使用情况

\subsection{外部可行性分析}
经济、技术、社会可行性分析

\vspace*{\baselineskip}

\section*{建议进度安排:}

\vspace*{\baselineskip}

\begin{description}
	\item[第1-2周] 查阅相关资料,调查国内外发展状况,写调研报告
	\item[第3周] 完成调研报告,查阅外文翻译资料,完成外文翻译
	\item[第4-5周] 业务建模:愿景、业务用例图、现状业务序列图、改进业务序列图
	\item[第6-7周] 系统需求:系统用例图、系统用例规约、系统活动图
	\item[第8-10周] 分析设计:类图、序列图、状态机图、数据表、交互界面(选)
	\item[第11周] 完成《系统设计方案》
	\item[第12周] 中期考核
	\item[第13周] 开发环境与开发工具选择
	\item[第14-18周] 软件编程,完成软件系统
	\item[第19周] 软件测试
	\item[第20周] 软件验收
	\item[第21周] 完成《软件使用说明书》
	\item[第22周] 完成《综合毕业设计》
	\item[第23周] 答辩
	\item[第24周] 资料归档
\end{description}

% ==================非实引文献===================
% 使用\nocite{}处理
\nocite{Tork:2022bef}\nocite{E598:1974sol}



% =======================以下部分用于导入参考文献(自动生成)==========================
 \printbibliography[title=参考文献,heading=bibintoc] % 导入参考文献