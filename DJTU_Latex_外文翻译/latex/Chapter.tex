% -*- coding=utf-8 -*-

%  模版标题——外文翻译
\begin{center}
	{\zihao{3} \heiti 外文翻译} 
\end{center}

 \vspace*{\baselineskip}

% -------------------- 模版操作区--------------------------------
\begin{center}
	% 在此处输入译文文献题目
	{\zihao{3} \heiti Container port drayage operations and management: Past and future}
\end{center}

\section{Introduction}

Intermodal container transportation is a series involving at least two transportation modes to distribute containers to their destinations with the transference between modes at the intermodal terminals (Crainic and Kim, 2007). The combination of different transportation modes can achieve optimal balance between cost and time, thereby increasing the transportation flexibility of supply chain. The long-haul intermodal container transportation by sea mainly consists of liner shipping and container port drayage services, and these two services interact at container terminals. Nevertheless, the large container terminals are nowadays placed under pressure to deal with the increasing flow of containers, as shown in Fig. 1 (Council, 2020). As a consequence, container trucks could be delayed significantly within and near to terminals. As the main haulage, container liner shipping has drawn much attention to its operations and management and that derives several comprehensive reviews (Meng et al., 2014; Vejvar et al., 2020). However, the critical literature reviews on container port drayage operations and management are absent. 

Container port drayage is an important part of realizing door-to-door service in container transportation by sea. Container drayage operations and management connect factories and ports, and solve the first/last-mile problem in the container logistics industry. Although container drayage only accounts for a small proportion of transport distance, the cost accounts for a large percentage between 25\% and 40\% (Macharis and Bontekoning, 2004). According to the Intermodal Association of North America, drayage services are split into 6 categories: inter-carrier drayage, intra-carrier drayage, expedited drayage, pier drayage, door-to-door drayage and shuttle drayage.

Due to the different requirements for the number and size of containers in various countries and regions, an imbalance between the supply and demand of empty containers exists. Such imbalance is usually reduced by container repositioning (Zhang et al., 2009, 2010; Sterzik and Kopfer, 2013). However, with the emergence of COVID-19, a more serious imbalance forms between supply and demand for empty containers. COVID-19 is a global disease that has rapidly shaken the world economy and has a major impact on the development of the shipping industry (Yazir et al., 2020). It is likely that supply chains in this industry will be subject to great uncertainties for the foreseeable future. The detrimental impact of uncertainties can be significantly mitigated by flexibility in seaport logistics (Russell et al., 2020). For example, to manage empty containers at the customers’ locations, Legros et al. (2019) explores an effective time-based two-thresholds strategy to limit inessential movement of empty containers in the hinterland. This affords greater potential for direct redeployment, thereby reducing container repositioning costs and also improving the use of containers. However, the realization of comprehensive flexibility is not simple, as it involves the interaction of multiple stakeholders. Early container drayage services mainly involved the decision of container drayage operators, with the objective of maximizing economic benefits subject to limited resources. As information cannot be shared among stakeholders quickly, the flexibility is quite poor. Fortunately, with the development of new technology, business patterns have changed, affording greater opportunity to reduce negative impact.

\subsection{Main stakeholders}

Container port drayage provides container pickup and delivery services between terminals and customers. The main stakeholders of container port drayage nowadays include: container drayage operators, container port operators, e-commerce platform operators, and customers or shippers. Container drayage operators play a vital role in the container port drayage operations. They are mainly concerned with the economic benefit maximization by making better use of various resources to fulfill the requests of customers. The target of container port operators is efficiently to manage the operations in terminal, thereby alleviating traffic problems for trucks, including gate/yard congestion. E-commerce platform operators can match the truck supply and container drayage demand more effectively, thereby greatly increasing service capabilities and improving service quality. They are relatively new among the main stakeholders. Customer considerations in all aspects are important. However, as customers are not involved in the decision making of container port drayage operations and management, they are more of a factor in the analysis of other issues, e.g., the demand function. The interactions among the stakeholders also directly affect the operations and management of container port drayage. Therefore, in any case, inefficient management of the drayage operations could lead to delays in shipments and congestion by the terminal and, thus, low overall productivity (Song et al., 2017).

\subsection{Objectives and contributions}

The goal of this study is critically to review the existing studies on container port drayage operations and management, and to highlight the findings of the current literature in various respects and the impact of these on various stakeholders from the perspective of operations research and management science. Through the analysis of these results, it identifies the potential problems for future research focus and directions, so as to promote the process of the development in practice of this industry. This is the first work that systematically analyzes the interactions among main stakeholders involved in container port drayage and the corresponding research results from the perspective of operations and management. It is of great significance both to academia and to industry.

Fig. 2 illustrates the terms that are frequently used by this paper. ‘‘Freight’’ represents the goods within a container. A loaded container with freight inside is called a ‘‘load’’. Tractor and trailer represent active and passive means respectively and, together, they form a truck. The mathematical notations used in the paper are listed in Appendix A. The rest of this paper is organized as follows: Section 2 presents the bibliometric analysis of the related literature. Sections 3–5 analyze the past and future of container port drayage operations and management from the perspective of container drayage operators, container port operators, e-commerce platform operators/customers, respectively. Section 6 investigates the interactions between main stakeholders. Section 7 identifies the potential issues for future research focus and directions. Section 8 summarizes the results and gives the conclusions.

\section{Bibliometric analysis}

We searched the relevant literature using the keyword ‘‘drayage’’ in the Web of Science (WoS) database, and collected a total of 175 papers from 1992 to 2021. Figs. 3–5 are drawn according to the analytical results of these papers. It should be noted here that (1) drayage related research actually appeared relatively late. Some early related research may not be called drayage, but we have also referred to some of this research; (2) Some studies may not put drayage in their titles and abstracts, leading to omission during retrieval. However, we have also included related papers based on the literature reviews in other papers. We have devoted ourselves to reviewing all the relevant studies and mainly analyzed the growth of these studies (Fig. 3), the areas of concerns (Fig. 4), and the changes in the directions of research (Fig. 5).

Fig. 3 demonstrates the trend of annual publication from 1992 to 2021. We can see that the publications per year before 2010 was few and the development was slow. After 2010, the number of publications has shown a steady upward trend. Even though the number in some years has decreased slightly, the number has been far greater than before. The causes are manifold. First of all, researchers are increasingly aware of the importance of improving the operations and management of container port drayage to the entire container shipping supply chain. Secondly, as people deepen the analysis of this problem, more research directions have been discovered, and with the innovation of technology and methodology, more and more aspects need to be considered, thereby inviting a vast increase in research.

Fig. 4 shows the treemap of top 20 research disciplines covered by the literature. According to the research discipline classification of WoS, the first five research disciplines are transportation, operations research management science, transportation science technology, engineering civil, and economics. As we can see, container port drayage has a major impact not only on urban transportation planning but also on social economy. It requires not only advances in science and technology, but also reasonable strategies for operations and management. Moreover, many related studies also involve the environmental concerns, which is also of great significance for the sustainable port city development.

Fig. 5 shows the overlay visualization of term co-occurrence map. It can illustrate the changes in the focus of research topics over time. Colors are rendered according to the terms occurring in the publications of normalized average year. Earlier research topics are colored blue and more recent ones in yellow. We can see that, the focus of early research is to discover and analyze the traffic congestion problems themselves. Then it gradually transforms to the development and promotion of scheduling solutions. Later on, researchers pay more attention to the improvement of service quality and the impact on the environment to achieve sustainable development.  

We have further analyzed those representative works, and we present our results from the perspective of different stakeholders in the subsequent sections.

\section{引言}

多式联运集装箱运输是一系列涉及至少两种运输方式,在多式联运码头的模式之间转移集装箱到目的地(Crainic和Kim,2007年)。不同运输方式的结合可以在成本和时间之间实现最佳平衡,从而提高供应链的运输灵活性。海上长途多式联运集装箱运输主要包括班轮运输和集装箱港口运输服务,这两种服务在集装箱码头互动。然而,如图所示,如今大型集装箱码头面临应对不断增加的集装箱流量的压力。1(理事会,2020年)。因此,集装箱卡车可能会在码头内和码头附近大量延误。作为主要运输,集装箱班轮运输引起了人们对其运营和管理的大量关注,并由此产生了一些全面的审查(Meng等人,2014年;Vejvar等人,2020年)。然而,没有关于集装箱港口运营和管理的批判性文献评论。

集装箱港口运输是实现集装箱海上运输门到门服务的重要组成部分。集装箱运输运营和管理连接工厂和港口,解决集装箱物流行业的第一/最后一英里问题。虽然集装箱运输只占运输距离的一小部分,但成本占25\%和40\%之间的很大比例(Macharis和Bontekoning,2004年)。根据北美多式联运协会,疏通服务分为6类:承运人间疏通、承运承内疏通、加急疏通、码头疏通、门到门疏通和穿梭疏通。

由于不同国家和地区对集装箱数量和尺寸的要求不同,空集装箱的供需存在不平衡。这种不平衡通常通过容器重新定位来减少(Zhang等人,2009年,2010年;Sterzik和Kopfer,2013年)。然而,随着新冠肺炎的出现,空集装箱的供求关系形成了更严重的不平衡。新冠肺炎是一种全球性疾病,它迅速动摇了世界经济,并对航运业的发展产生了重大影响(Yazir等人,2020年)。在可预见的未来,该行业的供应链可能会受到很大的不确定性的影响。海港物流的灵活性可以显著减轻不确定性的不利影响(Russell等人,2020年)。例如,为了管理客户所在地的空集装箱,Legros等人(2019)探索了一种有效的基于时间的双阈值策略,以限制空集装箱在腹地的非必要移动。这为直接重新部署提供了更大的潜力,从而降低了集装箱重新定位成本,也改善了集装箱的使用。然而,实现全面灵活性并不简单,因为它涉及多个利益攸关方的互动。早期的集装箱运输服务主要涉及集装箱运输运营商的决定,目的是在有限的资源下实现经济效益最大化。由于信息无法在利益相关者之间快速共享,灵活性相当差。幸运的是,随着新技术的发展,商业模式发生了变化,为减少负面影响提供了更大的机会。

\subsection{主要利益相关者}

集装箱港口运输在码头和客户之间提供集装箱取货和送货服务。如今,集装箱港口运输的主要利益相关者包括:集装箱运输运营商、集装箱港口运营商、电子商务平台运营商以及客户或托运人。集装箱疏淣操作员在集装箱港口混雷操作中发挥着至关重要的作用。他们主要关注的是通过更好地利用各种资源来满足客户的要求来最大化经济效益。集装箱港口运营商的目标是有效地管理码头的运营,从而缓解卡车的交通问题,包括登机口/货场拥堵。电子商务平台运营商可以更有效地匹配卡车供应和集装箱运输需求,从而大大提高服务能力,提高服务质量。在主要利益相关者中,他们相对较新。客户各方面的考虑都很重要。然而,由于客户不参与集装箱港口运营和管理的决策,他们更像是分析其他问题的一个因素,例如需求功能。利益相关者之间的互动也直接影响了集装箱港口的运营和管理。因此,无论如何,对延迟操作的管理效率低下可能会导致运输延迟和码头拥堵,从而导致整体生产力低下(Song等人,2017年)。

\subsection{目标和贡献}

本研究的目标是批判性地审查关于集装箱港口运输运营和管理的现有研究,并从运营研究和管理科学的角度强调当前文献在各个方面的发现以及这些研究对各个利益相关者的影响。通过对这些结果的分析,它确定了未来研究重点和方向的潜在问题,从而促进了该行业在实践中的发展进程。这是第一项从运营和管理角度系统分析涉及集装箱港口疏沕的主要利益相关者之间的相互作用以及相应的研究结果的工作。它对学术界和工业界都意义重大。

图2说明了本文经常使用的术语。“货物”代表集装箱内的货物。装有货物的集装箱被称为“装载”。拖拉机和拖车分别代表主动和被动手段,它们共同构成一辆卡车。论文中使用的数学符号列在附录A中。本文的其余部分组织如下:第2节介绍了相关文献的书目分析。第3-5节分别从集装箱码头运营商、集装箱港口运营商、电子商务平台运营商/客户的角度分析了集装箱港口码头运营和管理的过去和未来。第6节调查了主要利益相关者之间的互动。第7节确定了未来研究重点和方向的潜在问题。第8节总结了结果并给出了结论。

\section{文献计量分析}

图3显示了1992年至2021年的年度出版趋势。我们可以看到,在2010年之前,每年的出版物很少,而且发展缓慢。2010年后,出版物数量呈稳步上升趋势。尽管有些年这个数字略有下降,但这个数字比以前大得多。原因多种多样。首先,研究人员越来越意识到改善集装箱港口运营和管理对整个集装箱运输供应链的重要性。其次,随着人们对这个问题的深入分析,发现了更多的研究方向,随着技术和方法的创新,需要考虑越来越多的方面,从而吸引了研究的大幅增加。

图4显示了文献中涵盖的前20个研究学科的树图。根据WoS的研究学科分类,前五个研究学科是运输、运营研究管理科学、运输科学技术、工程土木和经济学。正如我们所看到的,集装箱港口的划淈不仅对城市交通规划,而且对社会经济也有重大影响。它不仅需要科学技术的进步,还需要合理的运营和管理策略。此外,许多相关研究还涉及环境问题,这对可持续的港口城市发展也具有重要意义。

图5显示了术语共发生图的叠加可视化。它可以说明研究课题重点随着时间的推移而发生的变化。颜色是根据标准化平均年出版物中出现的术语呈现的。早期的研究主题是蓝色的,最近的研究主题是黄色的。我们可以看到,早期研究的重点是发现和分析交通拥堵问题本身。然后,它逐渐转变为调度解决方案的开发和推广。后来,研究人员更加关注服务质量的提高和对环境的影响,以实现可持续发展。

我们进一步分析了这些代表性工作,并在后续章节中从不同利益相关者的角度介绍了我们的结果。


% 以下部分用于标注译文出处

\vspace*{\baselineskip}

\noindent
% 注意出处后边的}直接粘贴参考文献信息,不要空行
{\zihao{-4} \songti \bfseries 出处:} Chen R, Meng Q, Jia P. Container port drayage operations and management: Past and future[J]. Transportation Research Part E: Logistics and Transportation Review, 2022, 159: 102633.



















