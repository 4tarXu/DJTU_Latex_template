% -*- coding=utf-8 -*-
%===================模版操作区===========================

\chapter{绪论}

\section{系统开发背景}

\subsection{社会环境}

在当今信息化社会,大学生需要具备较强的协作知识建构能力,以适应快速变化的 社会环境和工作需求。协作知识建构能力不仅包括对知识的理解和掌握,还涉及到在团 队中与他人协作、交流和共享知识的能力。因此,培养大学生的协作知识建构能力成为 教育改革的重要目标之一。传统的教学模式往往注重知识的传授和学生的个体学习,而 忽视了学生之间的协作和互动。然而,在现代社会中,个体学习和竞争已经无法满足复 杂问题的解决和创新的需求。协作学习作为一种主要的教学方式,强调学生之间的合作、 交流和共享,能够有效地提高学生的协作知识建构能力\cite{__2025} 。

在传统教育评价体系中,一张试卷、一次考试成绩往往成为衡量学生学习成果的唯 一标准,这种 “一考定乾坤” 的终结性评价方式,如同用一张静态的照片去定义学生 整个学习阶段的成长,严重忽视了学生在学习过程中所付出的努力、经历的进步以及展 现的潜力。随着教育理念的不断革新,教育评价也亟需突破固有模式,过程性考核作为 一种关注学生成长轨迹的评价创新,逐渐成为教育领域的新焦点。

……

\subsection{系统发展和应用现状}

随着教育信息化的深入推进,学生成绩管理作为学校教学管理中的核心环节,其重 要性日益凸显。传统的手工录入、纸质存档和人工查询等成绩管理方式,不仅效率低下, 容易出错,而且难以满足现代教育对学生信息即时性、准确性和共享性的要求。成绩管 理系统的现状在国内外有所不同。 

国内……

国外……

……

\subsection{组织现实问题}

本系统开发的现实来源基于现实教学工作中存在的真实需求,通过现场调研,采访 了本学院的任课教师,了解到相关情况。

以本学院《最优化导论》课程为例,任课教师所采用的即是结合“动态分组”的过程性考核新模式,对授课方式与考核方式均进行了调整:拆班教学、过程性考核、新增个人考核系数。

在该考核模式下,由此为其带来了诸多难题,例如:

(1)基于“动态分组”这一前提下进行的考核模式,每一次课堂上关于学生的出勤与 考核,目前还只能限于用拍照留存这一形式作于记录。

(2)由于每次课堂上的分组及考核的内容不一样,因而学生每次获得的分数也不一样,与之而来的则是任课教师需要耗费极大的精力自己去手动整理众多的考核资料以及各种各样的成绩核算,目前对于上述资料信息的整理记录仅限于Excel 表格形式,见表 1。

\begin{figure}[htbp]
	\centering
	\includegraphics[width=\linewidth]{figure/截屏2025-11-30 16.58.49.png}
\end{figure}



(3)目前,任课教师每次上传的 HTML 考核内容都设置了一个结果图保存的功能,学 生保存结果后目前只能将结果图保存到本地,而任课教师对于该结果的记录依旧只有手 动收集整理图片这一形式,见图 1。

\begin{figure}[htbp]
	\centering
	\includegraphics[width=0.8\linewidth]{figure/截屏2025-11-30 17.01.09.png}
	\caption{小组随堂考核结果}
\end{figure}



随着教育信息化的深入推进,学生成绩管理作为学校教学管理中的核心环节,其重要性日益凸显。传统的手工录入、纸质存档和人工查询等成绩管理方式,不仅效率低下, 容易出错,而且难以满足现代教育对学生信息即时性、准确性和共享性的要求。成绩管 理系统的现状在国内外有所不同。

……

\section{系统开发意义}

\subsection{系统开发目标}

本系统拟设计开发一个学生课程成绩管理系统,应用于动态分组下的过程性考核课 程的资料存储与成绩核算,核心功能如下:

\begin{itemize}
	\item 课程管理——学生分组排座、小组出勤导入、基于 HTML 教学内容导入
	\item 学习管理——查阅分组情况、在线参加考核
	\item 成绩管理——考核成绩转存、成绩核算模块
\end{itemize}

……

\subsection{应用价值}

本系统的设计与实现旨在解决对于部分高校课程的基于“动态分组”的过程性考核 模式当中存在的各种难题,既能为学校及授课教师提供可采用的学生成绩管理系统,促 进更高效的教学和管理,又能为其它教学课程考核带来更好的示范性作用,推动学习与 应用。

……

\section{可行性分析}
\subsection{技术可行性}

技术可行性评估系统的技术需求、实现难度和兼容性,确保其基于现有技术可稳定 实现。

……

\subsection{管理可行性}
……
\subsection{社会可行性}
社会可行性评估系统对社会、用户和教育环境的影响,包括接受度、公平性和伦理 问题。

……
\subsection{经济可行性}
经济可行性评估系统的成本、收益和投资回报,重点关注其是否能在教育资源有限 的情况下实现可持续运营。

……

\chapter{业务建模}

\section{愿景}

基于面向对象分析与设计的方法,在业务建模工作流中,首先需要定位目标组织及 其愿景。 

……

\subsection{目标组织}
(组织概述、业务简介)

 ……

\subsection{涉众}

系统的涉众覆盖教学全链路相关方,其需求与系统功能强关联见表 2。

\begin{table}[htbp]
	\centering
    \small
    \caption{ ...... }
	\begin{tabular}{|>{\centering\arraybackslash}p{67pt}|>{\centering\arraybackslash}p{180pt}|>{\centering\arraybackslash}p{170pt}|}
		\hline
		涉众类型   & 核心需求                                   & 与系统的交互方式                         \\ \hline
		授课教师   & 减少手工分组、成绩核算工作量;提 升考核公平性;灵活调整分组规则与 考核内容 & 登录系统创建课程、设置分组策略、 导入考核内容、查看动态成绩报表 \\ \hline
		学生     & 公平参与考核;明确个人贡献与成绩 关联;便捷查询成绩与反馈          & 参与在线考核、查看分组结果、查阅 个人成绩明细          \\ \hline
		学校管理层  & 教学过程可追溯;考核数据支撑教学 质量分析;符合教育信息化政策要求      & 查看全校/学院考核统计报表(如分 组覆盖率、成绩分布)      \\ \hline
		教务处管理员 & 系统稳定运行;数据安全(学生信息、 成绩隐私);权限分级管理         & 维护用户权限、备份考核数据、处理 系统异常            \\ \hline
		技术开发团队 & 系统迭代需求收集;功能优化(如算 法效率、兼容性)              & 收集用户反馈,更新分组算法、修复 BUG、适配新教务系统     \\ \hline
	\end{tabular}
	\label{tab:my-table}
\end{table}

 ……

\subsection{组织负责人}

(根据组织涉众及业务,分析确定组织负责人) 

……

\subsection{愿景及改进目标}

(从组织负责人的视角,希望系统带来的改进目标及主要度量指标)
 
 ……

 \section{业务用例图}

(定义组织为外部执行者提供的价值——业务用例) 

……

\subsection{业务执行者}

(确定业务执行者,可能多个)
 
 ……

\subsection{业务用例}
(绘制业务用例图,并辅以文字描述)
 
 ……

\section{业务序列图}

\subsection{现状业务序列图}
(基于业务用例,构建现状业务序列图,并结合愿景分析存在的问题) 

……
\subsection{改进业务序列图}
(引入新系统,绘制改进业务序列图,说明改进效果)
 
 ……

\chapter{系统需求}
\section{系统用例图}

(从改进业务序列图,映射系统用例图,并辅以文字描述——多个)
 
 ……

\section{系统用例规约}
(针对每一个系统用例,整理用例规约,可用表格)

用例编号:用例名 

执行者: 

前置条件: 

后置条件: 

涉众利益: 

基本路径:

1. ××××

2. ××××

扩展路径:

2a. ××××:

2a1. ×××× 

补充约束: 

\begin{itemize}
    \item 字段列表
    \item 业务规则
    \item 质量需求
    \item 设计约束
\end{itemize}


\begin{table}[htbp]
	\centering
    \small
 	\caption{用例规约 xx}
	\begin{tabular}{|>{\centering\arraybackslash}p{60pt}|>{\centering\arraybackslash}p{80pt}|>{\centering\arraybackslash}p{80pt}|>{\centering\arraybackslash}p{150pt}}
		\hline
		用例编号   &         & 用例名     & \multicolumn{1}{c|}{}                     \\ \hline
		执行者    & \multicolumn{3}{c|}{}                                                                                                                 \\ \hline
		前置条件   & \multicolumn{3}{c|}{}                                                                                                                 \\ \hline
		后置条件   & \multicolumn{3}{c|}{}                                                                                                                 \\ \hline
		涉众利益   & \multicolumn{3}{c|}{}                                                                                                                 \\ \hline
		步骤     & 基本路径    & 扩展路径    & \multicolumn{1}{c|}{补充约束}                 \\ \hline
		1      &         &         & \multicolumn{1}{c|}{\makecell{字段列表:XXXXXX\\业务规则:XXXXXX\\质量需求:XXXXXX}} \\ \hline
		2      &         &         & \multicolumn{1}{c|}{}                     \\ \hline
		...... &         &         & \multicolumn{1}{c|}{}                     \\ \hline
	\end{tabular}
	\label{tab:my-table}
\end{table}

\chapter{分析设计}
\section{分析类图}
(根据系统用例图及系统用例规约,得到系统主要的边界类、控制类和实体类,绘 制包含关系的实体类图) 

……

\section{分析序列图}
(根据系统用例图、系统用例规约和类图,绘制表示系统交互活动的序列图) 

……

\section{分析活动图}
(根据系统用例图、系统用例规约和序列图,绘制系统主要操作的活动图) 

……

\section{数据表}
(根据包含关系的实体类图,构建数据表) 

……

\section{原型(选)}
(使用 Axure 等工具构建系统核心功能的原型,选做) 
……

\chapter{总结}
(对建模、分析和设计过程进行总结)
 
 \begin{itemize}
     \item 取得的成果
     \item 存在的问题及解决思路
 \end{itemize}
 

\nocite{__2025,__2025-1,_obe_2025,__2025-2,__2025-3,__2022,__2023,__2024,__2024-1,_python_2025,_pythonmysql_2023,_web_2023,jiaweia_design_2024,wang_experimental_2023}



% ====================== 分割线=========================
