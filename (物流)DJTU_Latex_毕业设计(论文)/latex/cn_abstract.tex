% -*- coding=utf-8 -*-
\chapter*{摘\quad  要}
\addcontentsline{toc}{chapter}{摘\quad  要}  % 添加到目录
\newcommand{\keywords}[1]{\text{\heiti #1}}
\vspace*{0.5\baselineskip}

本文给出了大连交通大学本科毕业论文的写作规范和排版格式要求。文中格式可作为编排毕业论文的格式模板,供本科生参考使用。

摘要部分说明:

标题“摘要”之间空两行,采用三号字、黑体、居中,与正文空一行。

单位制一律换算成国际标准计量单位制,除特别情况外,数字一律用阿拉伯数码。文中不允许出现插图。重要的表格可以写入。

每段落首行缩进2个汉字;或者手动设置成每段落首行缩进2个汉字。

摘要正文叙述本设计(论文)的主要内容。中文摘要采用小四、宋体,在400汉字左右。摘要正文后,列出3-5个主题词。“关键词:”是关键词部分的引导,不可省略。关键词请尽量用《汉语主题词表》等词表提供的规范词。关键词与摘要之间空一行。“关键词:”字体为黑体,五号,加粗,顶格;关键词词间用分号间隔,末尾不加标点,3-5个,宋体,五号。


\vspace*{1.0\baselineskip}
\noindent
{\fontsize{10.5pt}{12pt}\selectfont \keywords{关键词:} 写作规范;排版格式;本科毕业论文}% 关键词:} 之后填写中文关键词

\clearpage						% 跳到目录下一页
