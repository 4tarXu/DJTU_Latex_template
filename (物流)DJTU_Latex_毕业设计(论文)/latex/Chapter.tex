% -*- coding=utf-8 -*-
%===================模版操作区===========================

\chapter{正文格式说明}

正文需全面表述(1)研究的前提、假设和条件;(2)模型的建立,设计、实验方案的拟定;(3)基本概念和理论基础;(4)计算的主要方法和内容;(5)设计、实验方法及结果和意义的阐述;(6)理论在实际中的应用等。

\section{论文格式基本要求}

论文格式基本要求:

(1)页边距:上2.5cm,下2.5cm,左2.5cm、右2.5cm;页眉:1.5cm,页脚:1.75cm订;

(2)行  距:固定值20磅;

(3)页  码:页面底部居中、宋体、小五、数字1,2,3等;

(4)字  体:正文全部宋体,小四。不能使用格式刷

\section{论文页眉页脚的编排}

页眉:大连交通大学本科毕业设计(论文)(居中、小五、黑体)正文开始标注页眉。

页脚:正文开始标注页码,居中对齐,宋体,小五

\section{章节标题格式}

(1) 章标题选用模板中的样式所定义的“标题1”,手动设置成字体:黑体,居中,字号:三号,行距:固定值20磅,段前、段后均为0行。与正文空一行。每章另起一页。在输入章标题之后,按回车键,即可直接输入每章正文。

(2) 每节的节标题选用模板中的样式所定义的“标题2”,居左;手动设置成字体:黑体,居左,字号:四号,行距:固定值20磅,段前、段后均为0行。

(3) 节中的一级标题选用模板中的样式所定义的“标题3”,居左;手动设置成字体:黑体,居左,字号:四号,行距:固定值20磅,段前、段后均为0行。

\section{各章之间的分隔符设置}

各章之间应重新分页,使用“分节符”进行分隔。

设置方法:在“插入”菜单中选择“分节符类型”,在弹出的窗口中选择分节符类型为“下一页”,确定即可另起一页。

\section{正文中的编号}

正文中的图、表、附注、公式一律采用阿拉伯数字分章编号。

如图1-2,表2-3,附注4-5,式6-7等。如“图1.2”就是指本论文第1章的第2个图。文中参考文献采用阿拉伯数字根据全文统一编号,如文献[3],文献[3,4],文献[6~10]等,在正文中引用时用右上角标标出。

\section{参考文献}
参考文献的著录,在行文中引用的地方标号。一般以出现的先后次序编号,编号以方括号括起,放在右上角,如[1],[3~5],然后在“参考文献”一节中,按标号顺序一一说明文献出处。 

示例如下:(字体为五号、宋体)
 
期刊类:[序号] 作者1,作者2,……作者n.文章名[J].期刊名,出版年,卷次(期次):论文在刊物中的页码 

图书类:[序号] 作者1,作者2,……作者n.书名[M].出版地:出版者,出版年 

会议论文集:[序号] 作者1,作者2,……作者n.论文名[A].论文集名[C],出版地:出版者,出版年,论文在刊物中的页码.

\chapter{图标格式说明}

\section{表的格式说明}

\subsection{表的格式示例}

正文小四号宋体,首行缩进2字符,表采用三线表在正文中的常用格式如表2-1所示,请参考使用。

表名设置为宋体五号,居中,位于表上,表内文字设置为宋体五号,垂直居中。

\begin{table}[htbp]
\centering
\vspace{0.5\baselineskip}
\caption{示例表格}\label{tab:example}
\begin{tabular}{cc} % 两列左对齐
\toprule
本质 & 过程 \\
\midrule
途径或方法 & 规划、实施、控制 \\
目标 & 效率、成本效益 \\
活动或作业 & 流动与储存 \\
处理对象 & 原材料、在制品、产成品、相关信息 \\
范围 & 从原点(供应商)到终点(最终顾客) \\
目的或目标 & 适应顾客的需求(产品、功能、数量、质量、时间、价格) \\
\bottomrule
\end{tabular}
\end{table}

表名与上文正文,表格与下文正文均留有一行空格。如上表所示。

\subsection{表的格式描述}

(1) 表的绘制方法

表要用WORD绘制,不要粘贴。

(2) 表的位置

1) 表格居中排列。

2) 表格与下文应留一行空格。

(3) 表的版式

1) 表的大小尽量以一页的页面为限,不要超限,一旦超限要加续表。

(4) 表名的写法

1) 表名应当在表的上方并且居中。编号应分章编号,如表2-1、表2-2。

2) 表名与上文留一空行。

3) 表及其名称要放在同一页中,不能跨接两页。

4) 表内文字全文统一,设置为宋体,五号。

5) 中文表名设置为宋体,五号,且居中。

\section{图的格式说明}

图在正文中的格式示例如图2-1所示。\vspace{\baselineskip}

 \begin{figure}[htbp]
 \centering
 \includegraphics[width=0.8\textwidth]{figure/example.png} % 图片文件名
 \caption{示例图片标题} % 标题自动格式化为"图2-1 示例图片标题"
 \end{figure}

图名与上文正文,图与下文正文均留有一行空格。如上图所示。

\subsection{图的格式描述}

(1) 图的绘制方法

1) 插图、照片应尽量通过扫描粘贴进本文。

2) 简单文字图可用WORD直接绘制。

(2) 图的位置

1) 图居中排列。

2) 图与上文应留一行空格。

3) 图中若有附注,一律用阿拉伯数字和右半圆括号按顺序编排,如注1),附注写在图的下方。

(3) 图的版式

1) “设置图片格式”的“版式”为“上下型”或“嵌入型”,不得“浮于文字之上”。

2) 图的大小尽量以一页的页面为限,不要超限,一旦超限要加续图。

(4) 图名的写法

1) 图名居中并位于图下,编号应分章编号,如图2-1、图2-2。

2) 图名与下文留一空行。

3) 图及其名称要放在同一页中,不能跨接两页。

4) 图内文字清晰、美观。

5) 中文图名设置为宋体,五号,居中。

\chapter{章标题}

\section{第三章第一节题目}
\vspace{\baselineskip}
\subsection{第三章第一节一级题目}
\vspace{\baselineskip}
\section{第三章第一节题目}
\vspace{\baselineskip}
\subsection{第三章第一节一级题目}



\nocite{WXHK202322015},\nocite{CXYY202331031}

\nocite{王亚平2006}\nocite{Sjostrand:2006za}











% ====================== 分割线=========================
